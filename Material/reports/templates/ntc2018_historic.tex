\documentclass[{{ settings.font_size }}pt,{{ settings.page_size|lower }}paper]{article}

% ============================================================================
% Template: NTC 2018 Historic Buildings
% Relazione di Calcolo Strutturale - Edifici Storici
% Conforme a NTC 2018 Cap. 8 + Linee Guida Beni Culturali 2011
% ============================================================================

% Encoding e lingua
\usepackage[utf8]{inputenc}
\usepackage[italian]{babel}

% Grafica e tabelle
\usepackage{graphicx}
\usepackage{float}
\usepackage{booktabs}
\usepackage{tabularx}
\usepackage{longtable}
\usepackage{multirow}

% Layout pagina
\usepackage{geometry}
\geometry{
    {{ settings.page_size|lower }}paper,
    top=2.5cm,
    bottom=2.5cm,
    left=3cm,
    right=2cm
}

% Header e footer
\usepackage{fancyhdr}
\pagestyle{fancy}
\fancyhf{}
\fancyhead[L]{\small Edificio Storico - {{ metadata.project_location }}}
\fancyhead[R]{\small \thepage}
\fancyfoot[C]{\small {{ metadata.designer_name }} - Rev. {{ metadata.revision }}}
\renewcommand{\headrulewidth}{0.4pt}
\renewcommand{\footrulewidth}{0.4pt}

% Hyperlinks
\usepackage{hyperref}
\hypersetup{
    colorlinks=true,
    linkcolor=blue,
    pdftitle={{ metadata.project_name }} - Edificio Storico,
    pdfauthor={{ metadata.designer_name }},
}

% Matematica
\usepackage{amsmath}
\usepackage{amssymb}

% Colori
\usepackage{xcolor}
\definecolor{heritage}{RGB}{139,69,19}
\definecolor{historic}{RGB}{160,82,45}

% ============================================================================
% Documento
% ============================================================================

\begin{document}

% Frontespizio personalizzato per edifici storici
\begin{titlepage}
\centering

\vspace*{2cm}

{\Huge \textbf{ {{ metadata.project_name }} }}\\

\vspace{1cm}

{\Large Relazione di Calcolo Strutturale}\\
{\large Edificio Storico Vincolato}\\

\vspace{0.5cm}

{\normalsize Conforme a:}\\
{\small NTC 2018 - Capitolo 8: Costruzioni Esistenti}\\
{\small Linee Guida per la valutazione del rischio sismico\\del patrimonio culturale (2011)}\\

\vfill

\begin{tabular}{rl}
\textbf{Località:} & {{ metadata.project_location }} \\

\textbf{Indirizzo:} & {{ metadata.project_address }} \\

\textbf{Committente:} & {{ metadata.client_name }} \\
\textbf{Progettista:} & {{ metadata.designer_name }} \\
 & {{ metadata.designer_order }} \\
\textbf{Data:} & {{ metadata.report_date }} \\
\textbf{Revisione:} & {{ metadata.revision }} \\
\end{tabular}

\vfill

{\footnotesize \textit{Documento riservato - Uso interno}}

\end{titlepage}

% Indice

\tableofcontents
\clearpage


% ============================================================================
% 1. PREMESSA E INQUADRAMENTO
% ============================================================================

\section{Premessa e Inquadramento}

\subsection{Premessa}

{{ introduction }}

\subsection{Vincoli e Tutela}

L'edificio oggetto di intervento è sottoposto a vincolo di tutela ai sensi del D.Lgs. 42/2004 (Codice dei Beni Culturali e del Paesaggio).

Gli interventi proposti sono stati concordati con la Soprintendenza competente e rispettano i criteri di:
\begin{itemize}
    \item Minimo intervento
    \item Compatibilità con le caratteristiche storiche e architettoniche
    \item Reversibilità degli interventi
    \item Utilizzo di materiali e tecniche tradizionali ove possibile
\end{itemize}

% ============================================================================
% 2. DESCRIZIONE EDIFICIO STORICO
% ============================================================================

\section{Descrizione dell'Edificio Storico}

\subsection{Caratteristiche Generali}

{{ building_description.text }}

\subsection{Inquadramento Storico}


\textbf{Epoca di costruzione}: {{ building_description.construction_period }}

L'edificio rappresenta un esempio significativo dell'architettura del periodo, con caratteristiche costruttive tipiche dell'epoca.


\subsection{Sistema Strutturale Esistente}

Il sistema strutturale è costituito da:
\begin{itemize}
    \item Murature portanti in {{ building_description.type|lower }}
    \item Solai tradizionali (legno, voltine, etc.)
    \item Eventuali elementi strutturali di pregio (archi, volte, cupole)
\end{itemize}

% ============================================================================
% 3. LIVELLO DI CONOSCENZA (NTC 2018 §C8.5.4)
% ============================================================================

\section{Livello di Conoscenza}

\subsection{Indagini Eseguite}

Ai sensi della NTC 2018 §C8.5.4, sono state condotte le seguenti indagini:

\begin{table}[H]
\centering
\caption{Livelli di approfondimento delle indagini}
\begin{tabular}{lc}
\toprule
\textbf{Tipologia Indagine} & \textbf{Livello} \\
\midrule
Geometria strutturale & Esteso \\
Dettagli costruttivi & Esteso \\
Proprietà dei materiali & Esteso \\
\bottomrule
\end{tabular}
\end{table}

\subsection{Livello di Conoscenza e Fattore di Confidenza}

In base alle indagini condotte, si è conseguito il seguente livello di conoscenza:

\begin{itemize}
    \item \textbf{Livello di Conoscenza (LC)}: LC2 - Conoscenza Adeguata
    \item \textbf{Fattore di Confidenza (FC)}: 1.20
\end{itemize}

Il fattore di confidenza FC viene applicato alle resistenze dei materiali secondo:
\begin{equation}
f_d = \frac{f_k}{\gamma_M \cdot FC}
\end{equation}

% ============================================================================
% 4. NORMATIVA DI RIFERIMENTO
% ============================================================================

\section{Normativa di Riferimento}

La presente relazione è redatta in conformità alle seguenti normative specifiche per edifici esistenti:

\begin{enumerate}

    \item {{ code }}

\end{enumerate}

% ============================================================================
% 5. CARATTERIZZAZIONE MATERIALI ESISTENTI
% ============================================================================

\section{Caratterizzazione dei Materiali Esistenti}

\subsection{Murature}


\begin{table}[H]
\centering
\caption{Proprietà materiali - Valori caratteristici ridotti per FC}
\begin{tabular}{lccc}
\toprule
\textbf{Materiale} & \textbf{$f_{mk}$} & \textbf{$\tau_{0k}$} & \textbf{E} \\
 & \textbf{[MPa]} & \textbf{[MPa]} & \textbf{[MPa]} \\
\midrule

{{ mat.name[:40] }} &
{{ "%.2f"|format(mat.f_k) }}-- &
-- &
{{ "%.0f"|format(mat.E) }}-- \\

\bottomrule
\end{tabular}
\end{table}

\textit{Nota}: I valori riportati sono già ridotti del fattore di confidenza FC = 1.20.


% ============================================================================
% 6. ANALISI STRUTTURALE EDIFICI ESISTENTI
% ============================================================================

\section{Analisi Strutturale}

\subsection{Metodi di Analisi Adottati}

Per la valutazione della sicurezza strutturale sono stati utilizzati i seguenti metodi:

\begin{itemize}
    \item \textbf{Analisi limite} per archi e volte (Metodo di Heyman)
    \item \textbf{Analisi cinematica lineare} per meccanismi locali (NTC §C8.7.1.3)
    \item \textbf{Analisi globale} agli elementi finiti
\end{itemize}

\subsection{Analisi di Archi e Volte}


\subsubsection{Metodo di Heyman - Analisi Limite}

L'analisi degli archi e delle volte è stata condotta secondo il metodo dell'analisi limite di Heyman, basato sulle seguenti ipotesi:
\begin{enumerate}
    \item Assenza di resistenza a trazione ($\sigma_t = 0$)
    \item Resistenza a compressione infinita
    \item Impossibilità di scorrimento tra i conci
\end{enumerate}

Il coefficiente di sicurezza geometrico è definito come:
\begin{equation}
FS = \frac{t_{actual}}{t_{min}}
\end{equation}

dove $t_{min}$ è lo spessore minimo per garantire l'equilibrio.

\textbf{Risultati}: I coefficienti di sicurezza calcolati risultano superiori a 1.5 per tutti gli archi e volte analizzate.


\subsection{Meccanismi Locali di Collasso}

Sono stati identificati e analizzati i seguenti meccanismi locali potenziali:
\begin{itemize}
    \item Ribaltamento semplice di parete
    \item Flessione verticale
    \item Ribaltamento di cantonale
\end{itemize}

% ============================================================================
% 7. INTERVENTI DI CONSOLIDAMENTO
% ============================================================================

\section{Interventi di Consolidamento Proposti}

\subsection{Criteri di Intervento}

Gli interventi sono progettati secondo i seguenti criteri:
\begin{itemize}
    \item \textbf{Compatibilità}: Utilizzo di materiali compatibili con quelli esistenti
    \item \textbf{Reversibilità}: Preferenza per interventi reversibili
    \item \textbf{Minimo intervento}: Limitazione agli interventi strettamente necessari
    \item \textbf{Durabilità}: Garanzia di durabilità nel tempo
\end{itemize}

\subsection{Rinforzi con Materiali Compositi (FRP/FRCM)}


Per il consolidamento delle murature sono previsti rinforzi in:
\begin{itemize}
    \item FRCM (Fabric Reinforced Cementitious Matrix) - Preferito per compatibilità e traspirabilità
    \item Connettori in FRP per cucitura lesioni
\end{itemize}

I rinforzi sono dimensionati secondo CNR-DT 215/2018 per FRCM.


\subsection{Tirantature Metalliche}

Sono previste tirantature metalliche in corrispondenza di:
\begin{itemize}
    \item Contrasto spinte di archi e volte
    \item Collegamento tra pareti ortogonali
    \item Collegamento pareti-solaio
\end{itemize}

% ============================================================================
% 8. VERIFICHE DI SICUREZZA
% ============================================================================

\section{Verifiche di Sicurezza}

\subsection{Obiettivi della Valutazione}

Ai sensi della NTC 2018 §8.3, la valutazione della sicurezza è finalizzata a:
\begin{itemize}
    \item Stabilire se la costruzione ha un livello di sicurezza adeguato
    \item Definire gli eventuali interventi necessari
    \item Valutare l'efficacia degli interventi proposti
\end{itemize}

\subsection{Risultati delle Verifiche}


\begin{longtable}{p{3cm}p{2.5cm}cccp{2cm}}
\caption{Verifiche strutturali - Stato attuale e post-intervento} \\
\toprule
\textbf{Elemento} & \textbf{Verifica} & \textbf{$\zeta_E$} & \textbf{Pre} & \textbf{Post} & \textbf{Esito} \\
 & & & \textbf{int.} & \textbf{int.} & \\
\midrule
\endfirsthead

\toprule
\textbf{Elemento} & \textbf{Verifica} & \textbf{$\zeta_E$} & \textbf{Pre} & \textbf{Post} & \textbf{Esito} \\
\midrule
\endhead

\bottomrule
\endlastfoot


{{ verif.element[:20] }} & {{ verif.type[:18] }} & {{ "%.2f"|format(verif.ratio) }} & -- & OK &

\textcolor{green}{VERIF.}

\textcolor{red}{N.V.}
 \\

\end{longtable}

\textit{Legenda}: $\zeta_E$ = Rapporto domanda/capacità (deve essere $\leq 1.0$ per verifica soddisfatta)


% ============================================================================
% 9. ELABORATI GRAFICI
% ============================================================================


\section{Elaborati Grafici e Fotografici}


\begin{figure}[H]
\centering
\includegraphics[width=0.8\textwidth]{figures/figure_{{ fig.number }}.pdf}
\caption{{{ fig.caption }}}
\end{figure}



% ============================================================================
% 10. CONCLUSIONI
% ============================================================================

\section{Conclusioni}

{{ conclusions }}

\subsection{Conformità alla Normativa}

Gli interventi proposti garantiscono:
\begin{itemize}
    \item Conformità alle NTC 2018 - Capitolo 8
    \item Rispetto delle Linee Guida per i Beni Culturali
    \item Compatibilità con il valore storico-artistico dell'edificio
\end{itemize}

\vspace{2cm}

\begin{flushright}
Il Progettista Strutturale\\
\vspace{1cm}
{{ metadata.designer_name }}\\
{{ metadata.designer_order }}\\
\vspace{1.5cm}
\_\_\_\_\_\_\_\_\_\_\_\_\_\_\_\_\_\_\_\_\_\_\_\_\_\_\_\_\_\_
\end{flushright}

\vspace{2cm}

\noindent
\textit{Il presente documento è stato redatto in conformità alle vigenti disposizioni in materia di tutela del patrimonio culturale e alle norme tecniche per le costruzioni.}

\end{document}
