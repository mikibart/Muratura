\documentclass[{{ settings.font_size }}pt,{{ settings.page_size|lower }}paper]{article}

% ============================================================================
% Template: NTC 2018 Standard
% Relazione di Calcolo Strutturale
% Conforme a NTC 2018 §10.1
% ============================================================================

% Encoding e lingua
\usepackage[utf8]{inputenc}
\usepackage[italian]{babel}

% Grafica e tabelle
\usepackage{graphicx}
\usepackage{float}
\usepackage{booktabs}
\usepackage{tabularx}
\usepackage{longtable}
\usepackage{multirow}
\usepackage{array}

% Layout pagina
\usepackage{geometry}
\geometry{
    {{ settings.page_size|lower }}paper,
    top=2.5cm,
    bottom=2.5cm,
    left=3cm,
    right=2cm,
    headheight=14pt,
    footskip=1cm
}

% Header e footer
\usepackage{fancyhdr}
\pagestyle{fancy}
\fancyhf{}
\fancyhead[L]{\small {{ metadata.project_name[:50] }}}
\fancyhead[R]{\small \thepage}
\fancyfoot[C]{\small {{ metadata.designer_name }} - {{ metadata.report_date }}}
\renewcommand{\headrulewidth}{0.4pt}
\renewcommand{\footrulewidth}{0.4pt}

% Hyperlinks
\usepackage{hyperref}
\hypersetup{
    colorlinks=true,
    linkcolor=blue,
    filecolor=magenta,
    urlcolor=cyan,
    pdftitle={{ metadata.project_name }},
    pdfauthor={{ metadata.designer_name }},
}

% Matematica
\usepackage{amsmath}
\usepackage{amssymb}

% Liste personalizzate
\usepackage{enumitem}

% Colori
\usepackage{xcolor}
\definecolor{ntcblue}{RGB}{0,51,153}
\definecolor{ntcgray}{RGB}{100,100,100}

% ============================================================================
% Informazioni Documento
% ============================================================================

\title{
    \vspace{-2cm}
    \begin{center}
    \includegraphics[width=0.15\textwidth]{{{ settings.logo_path }}}
    \end{center}
    \vspace{0.5cm}
    {\Huge \textbf{ {{ metadata.project_name }} }}\\
    \vspace{0.5cm}
    {\Large Relazione di Calcolo Strutturale}\\
    \vspace{0.3cm}
    {\large Conforme a NTC 2018 - D.M. 17/01/2018}
}

\author{
    \textbf{Progettista Strutturale}\\
    {{ metadata.designer_name }}\\
    {{ metadata.designer_order }}\\
    
    {{ metadata.designer_address }}
    
}

\date{{{ metadata.report_date }} - Rev. {{ metadata.revision }}}

% ============================================================================
% Documento
% ============================================================================

\begin{document}

% Frontespizio
\maketitle
\thispagestyle{empty}

\vfill

\begin{center}
\begin{tabular}{|l|p{10cm}|}
\hline
\textbf{Progetto} & {{ metadata.project_name }} \\
\hline
\textbf{Località} & {{ metadata.project_location }} \\

\hline
\textbf{Indirizzo} & {{ metadata.project_address }} \\

\hline
\textbf{Committente} & {{ metadata.client_name }} \\

\hline
 & {{ metadata.client_address }} \\

\hline
\textbf{Data} & {{ metadata.report_date }} \\
\hline
\textbf{Revisione} & {{ metadata.revision }} \\
\hline
\end{tabular}
\end{center}

\vfill

\clearpage

% Indice

\tableofcontents
\clearpage


% ============================================================================
% 1. PREMESSA
% ============================================================================

\section{Premessa}

{{ introduction }}

% ============================================================================
% 2. DESCRIZIONE DELL'OPERA
% ============================================================================

\section{Descrizione dell'Opera}

\subsection{Generalità}

{{ building_description.text }}

\subsection{Caratteristiche Edificio}

\begin{itemize}
    \item \textbf{Tipologia}: {{ building_description.type }}
    \item \textbf{Destinazione d'uso}: {{ building_description.usage }}
    
    \item \textbf{Numero di piani}: {{ building_description.num_stories }}
    
    
    \item \textbf{Epoca costruzione}: {{ building_description.construction_period }}
    
    
    \item \textbf{Altezza totale}: {{ "%.2f"|format(building_description.total_height) }} m
    
\end{itemize}

% ============================================================================
% 3. NORMATIVA DI RIFERIMENTO
% ============================================================================

\section{Normativa di Riferimento}

La presente relazione è redatta in conformità alle seguenti normative tecniche:

\begin{enumerate}[label=\arabic*.]

    \item {{ code }}

\end{enumerate}

% ============================================================================
% 4. CARATTERIZZAZIONE MATERIALI
% ============================================================================

\section{Caratterizzazione dei Materiali}

\subsection{Materiali Strutturali}


\begin{table}[H]
\centering
\caption{Materiali utilizzati - Proprietà caratteristiche}
\label{tab:materiali}
\begin{tabular}{lcccc}
\toprule
\textbf{Materiale} & \textbf{Tipo} & \textbf{$f_k$} & \textbf{E} & \textbf{$\gamma$} \\
 & & \textbf{[MPa]} & \textbf{[MPa]} & \textbf{[kN/m³]} \\
\midrule

{{ mat.name[:40] }} & {{ mat.type }} &
{{ "%.2f"|format(mat.f_k) }}-- &
{{ "%.0f"|format(mat.E) }}-- &
{{ "%.1f"|format(mat.density) }}-- \\

\bottomrule
\end{tabular}
\end{table}


% ============================================================================
% 5. AZIONI DI PROGETTO
% ============================================================================

\section{Azioni di Progetto}

\subsection{Carichi Permanenti}

\subsubsection{Pesi propri strutturali (G$_1$)}
{{ loads.permanent_G1.description }}

\subsubsection{Carichi permanenti portati (G$_2$)}
{{ loads.permanent_G2.description }}

\subsection{Carichi Variabili}

\subsubsection{Sovraccarichi variabili (Q)}
{{ loads.variable_Q.description }}


\subsection{Azione Sismica}

L'azione sismica è definita secondo NTC 2018 §3.2 per la località di {{ seismic_action.location }}.

\begin{table}[H]
\centering
\caption{Parametri sismici di base}
\begin{tabular}{ll}
\toprule
\textbf{Parametro} & \textbf{Valore} \\
\midrule
Accelerazione massima attesa $a_g$ & {{ "%.3f"|format(seismic_action.ag) }} g \\
Fattore amplificazione spettro $F_0$ & {{ "%.2f"|format(seismic_action.F0) }} \\
Periodo inizio spettro $T_C^*$ & {{ "%.2f"|format(seismic_action.Tc_star) }} s \\
Categoria sottosuolo & {{ seismic_action.soil_type }} \\
Categoria topografica & {{ seismic_action.topographic_category }} \\
Vita nominale $V_N$ & {{ seismic_action.nominal_life }} anni \\
Classe d'uso & {{ seismic_action.usage_class }} \\
Stato limite & {{ seismic_action.limit_state }} \\
\bottomrule
\end{tabular}
\end{table}


% ============================================================================
% 6. MODELLAZIONE STRUTTURALE
% ============================================================================

\section{Modellazione Strutturale}

\subsection{Software Utilizzato}

{{ modeling_assumptions.software }}

\subsection{Metodo di Analisi}

{{ modeling_assumptions.method }}

\subsection{Ipotesi di Calcolo}

\begin{itemize}

    \item {{ assumption }}

\end{itemize}

% ============================================================================
% 7. ANALISI STRUTTURALE
% ============================================================================

\section{Analisi Strutturale}

\subsection{Analisi Statica}


L'analisi statica è stata completata con successo.


\begin{itemize}
    \item \textbf{Spostamento massimo}: {{ "%.2f"|format(analysis_results.static_analysis.max_displacement) }} mm
    
    \item \textbf{Tensione massima}: {{ "%.2f"|format(analysis_results.static_analysis.max_stress) }} MPa
    
\end{itemize}




\subsection{Analisi Sismica}

L'analisi sismica è stata eseguita secondo NTC 2018 §7.3.


\begin{itemize}
    \item \textbf{Periodo fondamentale $T_1$}: {{ "%.3f"|format(analysis_results.seismic_analysis.fundamental_period) }} s
    
    \item \textbf{Fattore partecipazione modale}: {{ "%.2f"|format(analysis_results.seismic_analysis.participation_factor) }}
    
\end{itemize}



% ============================================================================
% 8. VERIFICHE STRUTTURALI
% ============================================================================

\section{Verifiche Strutturali}

\subsection{Criteri di Verifica}

Le verifiche sono condotte secondo NTC 2018 §4.5 per strutture in muratura, considerando:
\begin{itemize}
    \item Stati Limite Ultimi (SLU): resistenza e stabilità
    \item Stati Limite di Esercizio (SLE): deformabilità
\end{itemize}

\subsection{Riassunto Verifiche}


\begin{longtable}{p{3.5cm}p{3cm}cccl}
\caption{Verifiche strutturali - Riassunto} \\
\toprule
\textbf{Elemento} & \textbf{Verifica} & \textbf{$E_d$} & \textbf{$R_d$} & \textbf{$E_d/R_d$} & \textbf{Esito} \\
\midrule
\endfirsthead

\multicolumn{6}{c}%
{{\tablename\ \thetable{} -- continua}} \\
\toprule
\textbf{Elemento} & \textbf{Verifica} & \textbf{$E_d$} & \textbf{$R_d$} & \textbf{$E_d/R_d$} & \textbf{Esito} \\
\midrule
\endhead

\midrule
\multicolumn{6}{r}{{Continua nella pagina successiva}} \\
\endfoot

\bottomrule
\endlastfoot


{{ verif.element[:25] }} & {{ verif.type[:20] }} & {{ "%.2f"|format(verif.demand) }} & {{ "%.2f"|format(verif.capacity) }} & {{ "%.3f"|format(verif.ratio) }} &

\textcolor{green}{\textbf{VERIFICATO}}

\textcolor{red}{\textbf{NON VERIF.}}
 \\

\end{longtable}


% ============================================================================
% 9. ELABORATI GRAFICI
% ============================================================================


\section{Elaborati Grafici}


\begin{figure}[H]
\centering
\includegraphics[width=0.85\textwidth]{figures/figure_{{ fig.number }}.pdf}
\caption{{{ fig.caption }}}
\label{fig:{{ fig.number }}}
\end{figure}



% ============================================================================
% 10. CONCLUSIONI
% ============================================================================

\section{Conclusioni}

{{ conclusions }}

\vspace{1cm}

\begin{flushright}
Il Progettista Strutturale\\
\vspace{0.5cm}
{{ metadata.designer_name }}\\
{{ metadata.designer_order }}\\
\vspace{1cm}
\_\_\_\_\_\_\_\_\_\_\_\_\_\_\_\_\_\_\_\_\_\_\_\_\_\_\_\_\_\_
\end{flushright}

% ============================================================================
% APPENDICE (opzionale)
% ============================================================================


\clearpage
\appendix

\section{Tabulati di Calcolo}

In questa appendice sono riportati i tabulati dettagliati delle verifiche strutturali.



\end{document}
